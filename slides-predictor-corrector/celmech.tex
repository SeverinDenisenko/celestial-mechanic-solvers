\documentclass[10pt]{beamer}

\usetheme[progressbar=frametitle]{metropolis}
\usecolortheme{aggie}

\usepackage{appendixnumberbeamer}
\usepackage{booktabs}
\usepackage[scale=2]{ccicons}

\usepackage{pgfplots}
\usepgfplotslibrary{dateplot}

\usepackage{xspace}
\newcommand{\themename}{\textbf{\textsc{metropolis}}\xspace}

\title{Predictor-corrector solver for ODE}
\subtitle{Using Predictor-corrector solvers for 2- and 3- body problems}
% \date{\today}
\date{}
\author{Severin Denisenko}
\institute{Saint Petersburg State University}
% \titlegraphic{\hfill\includegraphics[height=1.5cm]{logo.pdf}}

\begin{document}

\maketitle

\begin{frame}{Table of contents}
  \setbeamertemplate{section in toc}[sections numbered]
  \tableofcontents%[hideallsubsections]
\end{frame}

\section[Concept]{Concept}

\begin{frame}{Concept}
    For predictor-corrector requared:
	\begin{itemize}
		\item First explicit method (predictor)
		\item Second impicit method (corrector)
		\item Predictor-corrector scheme (explaned next)
	\end{itemize}
\end{frame}

\begin{frame}{Predictor-corrector schemes}
\begin{itemize}
  \item Using the outcome of the explicit (predictor) method as an initial guess for the corrector (impicit) method
  \item Using the predictor result as a beginning value in an interative substitution in impicit method
\end{itemize}
\end{frame}

\begin{frame}{Predictor-corrector order of approximation}
    Interpolation Adams method has +1 order of approximation of enterpolation Adams method. Resulting precor pair will give interpolation's method approximation.
\end{frame}

\section{Interpolatin Adams method}

\begin{frame}{Second order}
  \begin{figure}
    \begin{tikzpicture}
      \begin{axis}[
        mlineplot,
        width=\textwidth,
        height=7cm,
        legend style={at={(0.8,0.8)},anchor=west},
        xmin=-0.5, xmax=1.5,
        ymin=-0.7, ymax=0.7,
        ]

        \addplot table [y=y, x=x, mark=none]{interpdt0dot01order2.dat};
        \addlegendentry{$dt = 0.01$}
        \addplot table [y=y, x=x, mark=none]{interpdt0dot001order2.dat};
        \addlegendentry{$dt = 0.001$}
        \addplot table [y=y, x=x, mark=none]{interpdt0dot0001order2.dat};
        \addlegendentry{$dt = 0.0001$}
        \addplot table [y=y, x=x, mark=none]{interpdt0dot00001order2.dat};
        \addlegendentry{$dt = 0.00001$}
      \end{axis}
    \end{tikzpicture}
  \end{figure}
\end{frame}

\begin{frame}{Second order}
  \begin{table}
    \caption{Second order method precision}
    \begin{tabular}{lr}
      \toprule
      dt & x on t=T\\
      \midrule
      0.01 & 1.01271285286200\\
      0.001 & 1.00001246611392\\
      0.0001 & 1.00000001277170\\
      0.00001 & 1.00000000000627\\
      \midrule
      exact & 1.0 \\
      \bottomrule
    \end{tabular}
  \end{table}
\end{frame}

\appendix

\begin{frame}[allowframebreaks]{References}
  \bibliographystyle{abbrv}
  \bibliography{celmech}
\end{frame}

\end{document}
